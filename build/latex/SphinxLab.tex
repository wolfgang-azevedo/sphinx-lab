%% Generated by Sphinx.
\def\sphinxdocclass{report}
\documentclass[letterpaper,10pt,english]{sphinxmanual}
\ifdefined\pdfpxdimen
   \let\sphinxpxdimen\pdfpxdimen\else\newdimen\sphinxpxdimen
\fi \sphinxpxdimen=.75bp\relax

\PassOptionsToPackage{warn}{textcomp}
\usepackage[utf8]{inputenc}
\ifdefined\DeclareUnicodeCharacter
% support both utf8 and utf8x syntaxes
\edef\sphinxdqmaybe{\ifdefined\DeclareUnicodeCharacterAsOptional\string"\fi}
  \DeclareUnicodeCharacter{\sphinxdqmaybe00A0}{\nobreakspace}
  \DeclareUnicodeCharacter{\sphinxdqmaybe2500}{\sphinxunichar{2500}}
  \DeclareUnicodeCharacter{\sphinxdqmaybe2502}{\sphinxunichar{2502}}
  \DeclareUnicodeCharacter{\sphinxdqmaybe2514}{\sphinxunichar{2514}}
  \DeclareUnicodeCharacter{\sphinxdqmaybe251C}{\sphinxunichar{251C}}
  \DeclareUnicodeCharacter{\sphinxdqmaybe2572}{\textbackslash}
\fi
\usepackage{cmap}
\usepackage[T1]{fontenc}
\usepackage{amsmath,amssymb,amstext}
\usepackage{babel}
\usepackage{times}
\usepackage[Bjarne]{fncychap}
\usepackage{sphinx}

\fvset{fontsize=\small}
\usepackage{geometry}

% Include hyperref last.
\usepackage{hyperref}
% Fix anchor placement for figures with captions.
\usepackage{hypcap}% it must be loaded after hyperref.
% Set up styles of URL: it should be placed after hyperref.
\urlstyle{same}
\addto\captionsenglish{\renewcommand{\contentsname}{Contents:}}

\addto\captionsenglish{\renewcommand{\figurename}{Fig.\@ }}
\makeatletter
\def\fnum@figure{\figurename\thefigure{}}
\makeatother
\addto\captionsenglish{\renewcommand{\tablename}{Table }}
\makeatletter
\def\fnum@table{\tablename\thetable{}}
\makeatother
\addto\captionsenglish{\renewcommand{\literalblockname}{Listing}}

\addto\captionsenglish{\renewcommand{\literalblockcontinuedname}{continued from previous page}}
\addto\captionsenglish{\renewcommand{\literalblockcontinuesname}{continues on next page}}
\addto\captionsenglish{\renewcommand{\sphinxnonalphabeticalgroupname}{Non-alphabetical}}
\addto\captionsenglish{\renewcommand{\sphinxsymbolsname}{Symbols}}
\addto\captionsenglish{\renewcommand{\sphinxnumbersname}{Numbers}}

\addto\extrasenglish{\def\pageautorefname{page}}

\setcounter{tocdepth}{1}



\title{Sphinx Lab Documentation}
\date{Jan 28, 2020}
\release{1.0}
\author{Wolfgang Azevedo}
\newcommand{\sphinxlogo}{\vbox{}}
\renewcommand{\releasename}{Release}
\makeindex
\begin{document}

\pagestyle{empty}
\sphinxmaketitle
\pagestyle{plain}
\sphinxtableofcontents
\pagestyle{normal}
\phantomsection\label{\detokenize{index::doc}}



\chapter{Sphinx Lab}
\label{\detokenize{lab/sphinx_lab:sphinx-lab}}\label{\detokenize{lab/sphinx_lab::doc}}
Bem-vindo ao SPHINX LAB a.k.a \sphinxstylestrong{sphinx\_lab}, desenvolvido por Wolfgang Azevedo.

\begin{sphinxadmonition}{important}{Important:}
Utilize este conteúdo para você organizar seus documentos e aprender uma nova ferramenta!
\end{sphinxadmonition}


\section{Tópico 1}
\label{\detokenize{lab/sphinx_lab:topico-1}}
Neste tutorial você vai aprender a instalar o Sphinx para:
\begin{itemize}
\item {} 
Documentar qualquer coisa que você queira;

\item {} 
Ajudar a sua equipe com o conteúdo \sphinxstyleemphasis{organizado} em tópicos;

\item {} 
Ajudar você na sua bagunça e ter em mãos os procedimentos rotineiros.

\end{itemize}


\subsection{Tópico 1.1 - Sub-tópicos}
\label{\detokenize{lab/sphinx_lab:topico-1-1-sub-topicos}}
Você pode criar sub-tópicos também….


\subsection{Tópico 1.2 - Imagens}
\label{\detokenize{lab/sphinx_lab:topico-1-2-imagens}}
Você pode adicionar imagens nos seus documentos.

\noindent\sphinxincludegraphics{{imagem}.png}


\subsection{Tópico 1.3 - Tabelas}
\label{\detokenize{lab/sphinx_lab:topico-1-3-tabelas}}
Você pode criar tabelas nos documentos.


\begin{savenotes}\sphinxattablestart
\centering
\begin{tabulary}{\linewidth}[t]{|T|T|}
\hline
\sphinxstyletheadfamily 
COL. A
&\sphinxstyletheadfamily 
COL. B
\\
\hline
LINHA 1
&
LINHA 1
\\
\hline
LINHA 2
&
LINHA 2
\\
\hline
\end{tabulary}
\par
\sphinxattableend\end{savenotes}


\subsection{Tópico 1.4 - Notas}
\label{\detokenize{lab/sphinx_lab:topico-1-4-notas}}
\begin{sphinxadmonition}{note}{Note:}
Voce pode inserir notas nos seus documentos….
\end{sphinxadmonition}


\chapter{Indices and tables}
\label{\detokenize{index:indices-and-tables}}\begin{itemize}
\item {} 
\DUrole{xref,std,std-ref}{genindex}

\item {} 
\DUrole{xref,std,std-ref}{modindex}

\item {} 
\DUrole{xref,std,std-ref}{search}

\end{itemize}



\renewcommand{\indexname}{Index}
\printindex
\end{document}